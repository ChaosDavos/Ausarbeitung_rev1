% !TEXroot=main.tex
\renewcommand\abstractname{Zusammenfassung} 
\begin{abstract}
	{\large\textbf{\abstractname:}} \newline
	Die Robotik nimmt ein immer größer werdendes Forschungsgebiet innerhalb der Wissenschaft ein. Roboter werden für unser alltägliches Leben immer wichtiger. Doch um sich in unser Umfeld, \zb als Staubsaugroboter, integrieren zu können, müssen diese auch in der Lage zu sein, in Umgebungen zu navigieren.
	In diesem Projekt haben wir uns mit der Kartierung und darauf aufbauenden Navigation innerhalb eines unbekannten Labyrinthes beschäftigt. Wie wir Menschen zum Finden eines Ortes eine Karte brauchen, so benötigen auch Roboter diese, um zu navigieren. Dazu wird das Robot-Operating-System genutzt, um den Roboter, einen Turtelbot, diese Aufgaben bewältigen zu lassen. Dazu kartiert dieser ein Labyrinth, kreiert eine Karte, durch welcher der Roboter schlussendlich eigenständig in jenem Labyrinth navigieren kann. 
\end{abstract}