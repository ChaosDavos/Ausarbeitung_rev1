% !TEXroot=main.tex
\section{Einleitung}
{
	Für uns Menschen ist das orientieren eine alltägliche Aufgabe. Ob in der Stadt, auf dem Weg nach Hause oder in einem Bürogebäude. Roboter, welche immer mehr Aufgaben automatisieren, haben das gleiche Problem. Auch diese müssen sich in Umgebungen zurechtfinden. Von Staubsaugrobotern zu autonom fahrenden Fahrzeugen, all diese Systeme müssen sich in einer verzweigten Umgebung zurechtfinden. Daher ist es auch in Wettbewerben beliebt, Aufgaben zu diesem Thema zu stellen. So auch im "`RoboCup" \copyright. Roboter müssen sich in einem Labyrinth zurechtfinden und verschiedene Aufgaben bewältigen. Daher kommt auch die Id\texttt{}ee für dieses Projekt. Ein Roboter soll sich in einem Labyrinth zurechtfinden - mit Hilfe einer selbst erstellten Karte. Basierend auf einer vereinfachten Version des ``RoboCup Rescue Maze`` ist das Ziel des Projekts die autonome Navigation, wobei jedoch keine Aufgaben wie beispielsweise die Rettung von Personen erfüllt werden. Dies geschieht basierend auf einem Turtlebot-Roboter, auf welchem das Robot-Operating-System operiert. Ziel ist, einen Roboter zu bauen, welcher in einem komplexen Labyrinth werden kann, welches auch Zyklen besitzt und somit nicht durch das Folgen einer Wand vollends durchlaufen werden kann.
}