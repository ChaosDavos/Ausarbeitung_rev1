% !TEXroot=main.tex

\section{Probleme}
{
	Natürlich gab es während des Projektes auch immer wieder Probleme und Hindernisse, auf welche im Folgenden eingegangen werden.
	\subsection{Bedienung}
	{
		Bei der Bedienung war vor allem die fehlende Dokumentation für den Controller und dessen Einstellung das Problem. Dadurch musste erst herausgefunden werden, auf welcher Gerätedatei (js0) dieser zu finden war. Gleichzeitig war die Konfiguration etwas kontraintuitiv, da die Steuerung durch zwei verschiedene Joysticks geschieht und gleichzeitig ein Knopf gedrückt werden muss.
	}

	\subsection{Kartierung}
	{
		Im Falle der Kartierung war vor allem das finden passender Parameter das Problem. Dabei war die Karte oft ungenau und es gab Verbindungen zwischen Pfaden, welche in Realität keine haben (die Karte verschmilzt). Dies musste durch die schnelle Aktualisierungsrate ausgeglichen werden.
		Ein weiteres kleineres Hindernis war, herauszufinden, dass Studenten den LiDAR-Sensor falsch herum auf den Roboter gebaut haben, wodurch der Roboter sich auf der Karte in eine andere Richtung bewegt hatte, als er in der Realität tat. Dies führte zu seltsamen Karten, konnte aber schlussendlich durch das Umbauen (Drehen) des LiDAR-Sensors gelöst werden.
	}
	
	\subsection{LiDAR-Sensor}
	{
		Der LiDAR-Sensor ist hat teilweise einige Probleme. Durch Reflexion des ausgestrahlten Lichtes im richtigen, meist steilen Winkel entstehen Punkte auf der Karte, welche als Hindernisse erkannt werden, aber keine sind. Diese finden sich auch auf der Costmap wieder. Da die Fehlmessungen konstant reproduzierbar waren und damit nicht zufällig erfolgten, konnte das Problem teilweise gelöst werden. Indem die Auflösung der Costmap verringert wurde. Dadurch beschreibt ein  Pixel einen größeren Bereich in der echten Welt und kleine Fehlmessungen werden ausradiert, da die benachbarten Punkte nicht als Hindernis erkannt werden. Trotzdem kam es immer wieder zu solchen Reflexionsfehlern.
	}

}
\newpage
\section{Zusammenfassung}
{
	Schon im heutigen Zeitalter ist die Navigation von Robotern in einer unbekannten Umgebung eine Sache der Möglichkeit. Mit Hilfe von Algorithmen, wie \zb GMapping, an deren Entwicklung viele Beteiligt waren, ist es möglich, eine Karte zu erstellen, indem man die Vorteile der Veröffentlichung von Open-Source-Paketen nutzt. Noch dazu bietet ROS ein ideales Grundprinzip, in welchem Komponenten schnell integriert und ausgewechselt werden können. So kann die Kompatibilität vieler Komponenten gewährleistet werden. Nicht zuletzt deshalb gibt es Pakete für den Turtlebot, welche diesen vereinfacht ansteuern lassen und eine Integration einer Pfadplanung, so \zb durch die move\textunderscore base, welche in vielen Robotern genutzt werden kann.
	Die nächsten Schritte für dieses Projekt wären möglicherweise eine Kamera, welche, ähnlich zu dem ``RoboCup-Maze`` Bilder bzw. Symbole an den Wänden des Labyrinthes erkennt und nach diesen handelt, was auch Möglichkeiten für kleinere Anbauten an den Roboter, wie \zb eine Taschenlampe oder einen Greifarm, liefert.
	%ROS-Komponenten in Grundlagen - ROS
}