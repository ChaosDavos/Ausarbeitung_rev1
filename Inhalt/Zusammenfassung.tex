% !TEXroot=main.tex
\newpage
\section{Zusammenfassung}
{
	Schon im heutigen Zeitalter ist die Navigation von Robotern in einer unbekannten Umgebung eine Sache der Möglichkeit. Mit Hilfe von Algorithmen,wie \zb GMapping, an deren Entwicklung viele Beteiligt waren ist es möglich, eine Karte zu erstellen, indem man die Vorteile der Veröffentlichung von Open-Source-Packeten nutzt. Noch dazu bietet ROS ein ideales Grundprinzip, in welchem Komponenten schnell integriert und ausgewechselt werden können. So kann die Kompatibilität vieler Komponenten gewährleistet werden. Nicht zuletzt deshalb gibt es Packete für den Turtlebot, welche diesen vereinfacht ansteuern lassen und eine Integrierung einer Pfadplanung, so \zb durch die move\textunderscore base, welche in vielen Robotern integriert werden kann.
	Die nächsten Schritte für dieses Projekt wären möglicherweise eine Kamera, welche, ähnlich zu dem ``RoboCup-Maze`` Bilder bzw. Symbole an den Wänden des Labyrinthes erkennt und nach diesen handelt, was auch Möglichkeiten für kleinere Anbauten an den Roboter, wie \zb eine Taschenlampe oder einen Greifarm, liefert.
	%ROS-Komponenten in Grundlagen - ROS
}