% !TEXroot=main.tex
\usepackage{enumitem}
\usepackage{polyglossia}
\setmainlanguage[babelshorthands=true]{german}
\setotherlanguage{english}

%\setkomafont{chapter}{\normalfont\Large\itshape} only koma classes (src...)

%Wichtig: danach
\usepackage{blindtext}
\usepackage{float}
\usepackage{listings}

\usepackage{amsmath}
\usepackage{mathtools}

\usepackage{graphicx}
%\usepackage{tabu}
\usepackage{booktabs}
\usepackage{multicol}
\usepackage{amsfonts} 

\usepackage{amssymb}    % Mathematische Symbole
\usepackage{booktabs}   % Korrekter Tabellensatz
\usepackage[hang,font={sf,footnotesize},labelfont={footnotesize,bf}]{caption}

\usepackage[% Seitenlayout anzeigen
left=2.5cm,
right=2.5cm,
top=2.5cm,
bottom=2cm,
%includeheadfoot
]{geometry}

%\usepackage{hyperref}




\usepackage[
detect-all,
locale=DE,
binary-units=true,
range-phrase={\,\dots\,},
]{siunitx}                % Größen mit Einheiten korrekt darstellen
\DeclareSIUnit{\Bit}{Bit}

% Hyperref sollte sehr spät eingebunden werden, da es viele
% Elemente modifiziert
\usepackage{hyperref}  % Hyperlinks
\usepackage{color}
% Farben definieren
\definecolor{linkblue}{RGB}{0, 0, 100}
\definecolor{linkblack}{RGB}{0, 0, 0}
\definecolor{comment}{RGB}{63, 127, 95}
\definecolor{darkgreen}{RGB}{14, 144, 102}
\definecolor{darkblue}{RGB}{0,0,168}
\definecolor{darkred}{RGB}{128,0,0}
\definecolor{javadoccomment}{RGB}{0,0,240}

% Einstellungen für das Hyperref-Paket
\hypersetup{
	colorlinks=true,      % Farbige links verwenden
	%    allcolors=linkblue,
	%    allcolors=black,
	linktoc=all,          % Links im Inhaltsverzeichnis
	linkcolor=linkblack,  % Querverweise
	citecolor=linkblack,  % Literaturangaben
	filecolor=linkblack,  % Dateilinks
	urlcolor=linkblack,   % URLs
	pdfstartpage=1
}

% Makros für typographisch korrekte Abkürzungen
\newcommand{\zb}[0]{z.\,B.\ }
\newcommand{\dahe}[0]{d.\,h.\ }
\newcommand{\ua}[0]{u.\,a.\ }
% Diese Zusammensetzungen bestehen jeweils aus zwei Wörtern.
% Daher muss ein Leerzeichen dazwischen stehen. Traditionell
% wird das durch einen kleinen Abstand realisiert.

\title{Kartierung eines und Navigation in einem Labyrinth}
\author{\textit{David Siekacz}}
\date{05. August 2022}

%\subtitle{Die AG-Dokumentation der Feld- Wald und Wiesenbiologie AG beinhaltet Forschungsergebnisse aus dem ganzen Labjahr}
\titlehead{Hochschule Mannheim}
\subject{Schriftliche Arbeit}
\publishers{Betreut durch Prof. Dr. Thomas Ihme}
%numeric, authortitle
%\usepackage[style=authoryear, backend=biber]{biblatex}
\usepackage[defernumbers=true,backend=biber,
isbn=false,                  % ISBN nicht anzeigen, gleiches geht mit nahezu
% allen anderen Feldern
autocite=inline,             % regelt Aussehen für \autocite
%      inline: Zitat in Klammern (\parancite)
%      footnote: Zitat in Fußnoten (\footcite)
%      plain: Zitat direkt ohne Klammern (\cite)
style=ieee,       % Legt den Stil für die Zitate fest
%      ieee: Zitate als Zahlen [1]
%      alphabetic: Zitate als Kürzel und Jahr [Ein05]
%      authoryear: Zitate Author und Jahr [Einstein (1905)]
hyperref=true,                  % Hyperlinks für Zitate
citestyle=numeric-comp,         % Fasst Nummernzitate zusammen [1-7, 10]
natbib=true                     % Komatibilität zu natbib
]{biblatex}  
\addbibresource{literaturreferenzen.bib}
